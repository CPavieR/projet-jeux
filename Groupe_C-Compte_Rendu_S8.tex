\documentclass[a4paper, 12pt]{article}
\usepackage[utf8]{inputenc}
\usepackage[T1]{fontenc}
\usepackage[frenchb]{babel}
\usepackage{libertine}
\usepackage[pdftex]{graphicx}
\usepackage{hyperref}




\begin{document}

\begin{titlepage}
  \begin{sffamily}
  \begin{center}
  

%ici on ajoute les images sur la même ligne (j'ai mis des images mais je pense que ça ne fonctionnera pas de votre côté pour l'instant)
\begin{minipage}[c]{.46\linewidth}
     \begin{center}
             \includegraphics[width=4cm]{logo-fds.png}
         \end{center}
   \end{minipage} \hfill
   \begin{minipage}[c]{.46\linewidth}
    \begin{center}
            \includegraphics[width=4cm]{logo-figure.png}
        \end{center}
 \end{minipage}
    \newline \newline
%ici il faudra voir pour séparer les images du titre (en l'état ils sont collés)
%on passe ici au début du code de la page de garde

    \textsc{\LARGE Faculté Des Sciences - Montpellier}\\[2cm]

    \textsc{\Large Compte rendu de la séance 8}\\[1.5cm]

    \HRule \\[0.4cm]
    { \huge \bfseries Projet CMI\\[0.4cm] }

    \HRule \\[2cm]
    \\[2cm]

    \begin{minipage}{0.4\textwidth}
      \begin{flushleft} \large
        Conrath Matthieu\\
        Robert Wendy\\
      \end{flushleft}
    \end{minipage}
    \begin{minipage}{0.4\textwidth}
      \begin{flushright} \large
       Pavie-Routaboul Clement\\
        Rebagliato Lucas\\
      \end{flushright}
    \end{minipage}

    \vfill
    {\large 27 mars 2022}
    \\
    {\url{https://github.com/CPavieR/projet-jeux}}

  \end{center}
  \end{sffamily}
\end{titlepage}

\newpage

%et à partir de là c'est le code pour le reste du doc


\section*{Introduction}
      Ce document est le compte rendu de notre avancé sur le projet du 18/03/2022 au 27/03/2022. Pendant cette semaine, nous avons ajouté une procédure permettant de soustraire au capital de l'entreprise un salaire pour chaque conducteur. Nous avons ainsi testé notre jeu et nous nous sommes aperçus que l'argent prélevé lors des contrats était trop important. Nous avons donc ajusté les sommes prélevées.

      D'une autre part, la partie graphique a subi de nombreuses modifications, la bibliothèque choisie (CSFML) n'étant pas pleinement fonctionnelle et disposant de peu de documentation, nous avons dû essayer avec une autre bibliothèque qui est SDL. Sur ce point là nous sommes donc forcés de repartir de zéro. 

\section{Objectifs accomplis et problèmes rencontrés}
     \begin{tabular}{|l|c|r|}
  \hline
  Objectif & Statut actuel & Problème(s) rencontré(s) \\
  \hline
   - Ajout des salaires & Terminés & Pas de problème \\
   - Test du jeu & En cours & Certain problèmes ont été trouvés mais \\
    &  & d'autres peuvent survenir \\
     - Aspect graphique & En cours & Changement du blibliothèque \\
  
  

  \hline
\end{tabular}
\section{Prochaines étapes}
    Pour l'instant le jeu semble fonctionner sans trop de problèmes. Nous devons encore travailler sur l'aspect graphique du jeu. Cependant, nous devrions clarifier l'affichage via le terminal en réalisant des menus par exemple. De plus, pour la suite, nous devrions aussi ajouter une fonction qui vérifie l'existence d'une sauvegarde (afin d'éviter de charger une sauvegarde qui n'existe pas), nous pourrions réécrire certaines parties du programme afin d'utiliser des allocations dynamiques plutôt que des allocations statiques et enfin, nous devrions supprimer les fonctions et procédures qui ne sont pas utilisées dans le code. 
    

\appendix  % On passe aux annexes
\section{Bibliographie}
\paragraph{1- }
(Livre physique) : \url{https://www.editions-hatier.fr/livre/prepabac-nsi-tle-generale-specialite-bac-2022-9782401075221}
\paragraph{2- }
\url{https://zestedesavoir.com/tutoriels/755/le-langage-c-1/1043_aggregats-memoire-et-fichiers/4279_structures/}

\end{document}
